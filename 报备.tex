\documentclass[UTF8]{ctexart}
\usepackage{amsmath, amsthm, amssymb, graphicx, amsfonts, indentfirst}
\usepackage{fancyhdr,color,framed,enumitem}
\usepackage[perpage]{footmisc}
\linespread{1.5}
\usepackage[letterpaper,top=2cm,bottom=2cm,left=3cm,right=3cm,marginparwidth=1.75cm]{geometry}
\usepackage[colorlinks,
            linkcolor=blue,
            anchorcolor=blue,
            citecolor=blue]{hyperref}

\ctexset{tocdepth=3}
\ctexset{subsection = {number=\arabic{subsection}}}       %subsection形式改为1,2,3,...
\title{活动报备}
\author{小笨熊}
\setlength{\parindent}{4pt}
\setlength{\headheight}{27pt}
\pagestyle{fancy}
\newcommand{\foo}{\hspace{-2.7pt}$\bullet$ \hspace{5pt}}
\begin{document}
\maketitle

写在前面:报备是一件很严肃的事情,报备有问题可能会给活动带来很大的负面影响,因此,希望每一位队长或活动负责认真填写报备表,仔细核对信息(时间、地点、人数等)是否准确,最好找随队帮忙找找bug。\underline{必须由队长或活动负责完成的内容已标记。}

\subsection{时间}

不能举办未报备的活动;原则上所有活动需要提前至少7天报备,但不需要提前过长时间。

\subsection[填写报备表]{\underline{填写报备表}}

由队长或活动负责自行填写,组织部负责在合适的时间催促。

报备表原表及参考文件见{\color{blue}\textit{1 报备}},请注意,前站正式都需要报备,但是可以整合为一张表。

\ref{起始}-\ref{结束}均为报备表中需要填写的内容。

\subsection[负责人意见]{\underline{负责人意见}}\label{起始}

理论上需要找社长签字,实际操作自行签字即可。

\subsection[指导老师意见]{\underline{指导老师意见}}

组织部将队长或活动负责拉入报备群,向卢老师介绍活动,并由队长或活动负责本人约定时间找卢老师签字(五四318教室或二体西运动科学实验室)。卢老师邮箱:\href{mailto:lfq@pku.edu.cn}{lfq@pku.edu.cn}。

\begin{itemize}[nosep]
    \item[] \textit{卢老师您好,我是10月30日大运河前站、11月6日大运河正式拉练队长任致远,21级物理学院本科生,以下是大运河前站和正式拉练活动方案。活动方案已同步发到您的邮箱,感谢老师的支持与指导!另外,有活动报备表需要您签字,请问您近期是否有时间呢?(发活动方案)}
\end{itemize}

\subsection{指导单位意见、团委意见}\label{结束}

在填写完前述所有内容后,将报备表交至五四体育中心北202-203房间之间的小篮子中,并在一旁的登记表中签字登记,由指导单位完成意见填写。

报备表的工作至此完成,其余部分的填写由组织部完成。

\subsection[个人责任书]{\underline{个人责任书}}

非外出活动无需个人责任书。

队长在出摊检车前将修改好的个人责任书交给拉秘打印,个人责任书模板请见{\color{blue}\textit{2 拉秘打印资料}}。周五出摊检车时每人签两份个人责任书(一份交给团委(新太阳127),一份协会留存),务必在周五16:30前交!


\subsection[活动总结]{\underline{活动总结}}

活动结束后,应尽快向卢老师提交活动总结,基本要求如下:
\begin{itemize}[nosep,left=4em]
    \item 不必很详细,参考往年总结结构即可, 包括【拉练简要概况】【安全\&纪律情况】【大队骑行状况】【总结】;
    \item 要有合照\&活动的照片;
    \item 不要涉及危险行为,如危险动作的照片、摔车的记录等;
    \item 务必尽快提交,否则会影响下次拉练。
\end{itemize}

\end{document}

