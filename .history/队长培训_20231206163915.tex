\documentclass[UTF8]{ctexart}
\usepackage{amsmath, amsthm, amssymb, graphicx, amsfonts, indentfirst}
\usepackage{fancyhdr,color,framed,enumitem}
\usepackage[perpage]{footmisc}
\linespread{1.5}
\usepackage[letterpaper,top=2cm,bottom=2cm,left=3cm,right=3cm,marginparwidth=1.75cm]{geometry}
\usepackage[colorlinks,
            linkcolor=blue,
            anchorcolor=blue,
            citecolor=blue]{hyperref}

\ctexset{tocdepth=3}
%\ctexset{subsubsection = {number=\arabic{subsubsection}}}       %subsubsection形式改为1,2,3,...
\title{队长培训}
\author{小笨熊}
\setlength{\parindent}{4pt}
\setlength{\headheight}{27pt}
\pagestyle{fancy}
\newcommand{\foo}{\hspace{-2.7pt}$\bullet$ \hspace{5pt}}
\begin{document}
\maketitle

\tableofcontents

\clearpage

\section{队长与随队}

\subsection{什么是一个好的拉练?}

\begin{itemize}[nosep,left=2em]
    \item 前期筹备得当,有一定预案;
    \item 拉练过程中出现问题得体解决;
    \item 职务人员任务完成好;
    \item 多数人拉练体验不错。
\end{itemize}

一些\textbf{误区}:\textit{回来时间早、时间点和预期完全相同、完全不出意外……}

\subsection{队长的定位}

\subsubsection{对外形象}

树立一定的威信,胸有成竹。

\subsubsection{基本要求}

在从筹备到总结的各个阶段,让自己成为最了解情况的人。
\begin{itemize}[nosep,left=2em]
    \item \textbf{前期准备} \ 了解每个(前站)职务人员的职务完成情况;
    \item \textbf{拉练途中} \ 基本对整个大队情况有所了解;
    \item \textbf{拉练结束} \ 知晓拉练中出现的问题,进行总结和反思。
\end{itemize}

\subsection{队长的职责}

\subsubsection{把控大局}
\begin{itemize}[nosep,left=2em]
    \item 拉练的全权负责人;
    \item 多数情况下做出的决策应当服从大局;
    \item 始终和大部分人在一起。
\end{itemize}

\subsubsection{培养前站}
\begin{itemize}[nosep,left=2em]
    \item 职务培训,告知每个职务人员需要做什么;
    \item 引导职务人员主动思考,如何更好地完成职务;
    \item 一些理念的传承(纪律,团队……)。
\end{itemize}

\subsubsection{任务分流}
\begin{itemize}[nosep,left=2em]
    \item 首先对整个拉练有完整的认知,可以自己在脑子中过一遍;
    \item 明确各个职务人员需要完成的任务,在职务培训中给职务人员下派任务;
    \item 在后续过程中对职务人员进行验收;
    \item 队长只需要下派任务和关注结果,不必关注具体细节(晚会组织,前旗助认路等)。
\end{itemize}

\subsubsection{严明纪律,维持秩序}

注意纪律、团队和安全意识的重要性。

维持纪律的几种方式
\begin{itemize}[nosep,left=4em]
    \item \textbf{队长讲话}:充分利用每一次可以讲话的时机,队长讲话时传递信息最有效的时机,包括但不限于准备会讲话、出发前队长讲话、解散前队长讲话;应注意提前写稿、精简、重点突出;
    \item 过程中秩序的维护:出现问题及时提醒;
    \item 罚跑、早饭前站:惩罚只是手段不是目的,要和会员讲清楚处罚的原因。
\end{itemize}

\subsection{随队理事}

\begin{itemize}[nosep,left=2em]
    \item 辅助、监督作用,帮助筹备拉练;
    \item 涉及重要决策或队长拿不准的问题,提供参考意见(职务,路线,突发状况,流程安排);
    \item 审核论坛发帖内容、群公告等面向全组内容;
    \item 无需盲从,和随队有争议时可询问更多老会员意见,无法达成共识时可按队长想法施行。
    
\end{itemize}

\section{前期准备}

\subsection{路书}

当次拉练的路书由队长全权确定,可以参考往年路线,可更改,综合考虑道路载荷、路况、风景、强度等,画出路书。

\subsection{报备}

写在前面:报备是一件很严肃的事情,报备有问题可能会给活动带来很大的负面影响,因此,希望每一位队长认真填写报备表,仔细核对信息(时间、地点、人数等)是否准确,最好找随队帮忙找找bug。\underline{必须由队长完成的内容已标记。}

\subsubsection{时间}

不能举办未报备的活动;原则上所有活动需要提前至少7天报备,但不需要提前过长时间。

\subsubsection[填写报备表]{\underline{填写报备表}}

由队长或活动负责自行填写,组织部负责在合适的时间催促。

报备表原表及参考文件见{\color{blue}\textit{1 报备}},请注意,前站正式都需要报备,但是可以整合为一张表。

\ref{起始}-\ref{结束}均为报备表中需要填写的内容。

\subsubsection[负责人意见]{\underline{负责人意见}}\label{起始}

理论上需要找社长签字,实际操作自行签字即可。

\subsubsection[指导老师意见]{\underline{指导老师意见}}

组织部将队长或活动负责拉入报备群,向卢老师介绍活动,并由队长本人约定时间找卢老师签字(五四318教室或二体西运动科学实验室)。卢老师邮箱:\href{mailto:lfq@pku.edu.cn}{lfq@pku.edu.cn}。

\begin{itemize}[nosep]
    \item[] \textit{卢老师您好,我是10月30日大运河前站、11月6日大运河正式拉练队长任致远,21级物理学院本科生,以下是大运河前站和正式拉练活动方案。活动方案已同步发到您的邮箱,感谢老师的支持与指导!另外,有活动报备表需要您签字,请问您近期是否有时间呢?(发活动方案)}
\end{itemize}

\subsubsection{指导单位意见、团委意见}\label{结束}

在填写完前述所有内容后,将报备表交至五四体育中心北202-203房间之间的小篮子中,并在一旁的登记表中签字登记,由指导单位完成意见填写。

报备表的工作至此完成,其余部分的填写由组织部完成。

\subsubsection[个人责任书]{\underline{个人责任书}}

队长在出摊检车前将修改好的个人责任书交给拉秘打印,个人责任书模板请见{\color{blue}\textit{2 拉秘打印资料}}。周五出摊检车时每人签两份个人责任书(一份交给团委(新太阳127),一份协会留存),务必在周五16:30前交!

\subsubsection[活动总结]{\underline{活动总结}}

活动结束后,应尽快向卢老师提交活动总结,基本要求如下:
\begin{itemize}[nosep,left=4em]
    \item 不必很详细,参考往年总结结构即可, 包括【拉练简要概况】【安全\&纪律情况】【大队骑行状况】【总结】;
    \item 要有合照\&活动的照片;
    \item 不要涉及危险行为,如危险动作的照片、摔车的记录等;
    \item 务必尽快提交,否则会影响下次拉练。
\end{itemize}

\subsection{往年经验}

\subsubsection{职务培训}

充分了解队伍中所有职务的职务内容,对各职务的了解程度达到自己能够胜任一个职务,可参考\href{https://chexie.net/bbs/content/?bid=7&tid=1062&p=1}{\textit{常规拉练职务培训}}、{\color{blue}\textit{5 扩展阅读}}中的{\color{blue}\textit{马丁《如何当好一个拉练队长》、霖《如何当好一个双日拉练队长》}}。

\subsubsection{阅读往年总结帖}

阅读对应拉练、相似路线、定位相似的队长总结帖和相应的执委会记录。

\subsubsection{和往年队长面谈}

建议队长先思考,再面谈。可以考虑路线、时间、人员安排、食宿地点等方面,以及拉练本身需要注意的点和自己疑惑的问题。

\subsubsection{文书类工作}

编写前站招募帖、前站公布帖、拉练报名帖、前站通知帖等。

注意,在论坛发帖时请勿附保险二维码,而是使用保险链接\url{https://cps.xiebao18.com/llb1000885/product/detail-103675-127748.html}。

注意,北京大学自行车协会纪律处罚条例于2022年10月修订,新链接为\url{https://www.chexie.net/bbs/content/?bid=1&tid=9260&p=1},请勿使用往年帖中的旧链接。

\section{时间线}

前站周
\scalebox{1}{
\begin{tabular}{r |@{\foo} l}
周一 & 发布前站招募帖\\
周二前& 发活动计划给卢老师\\
周一-周三 & 约饭和职务安排\\
周三/周四 & 公布前站,拉群\\
周四-周五 & 职务人员培训和验收\\
周五 & 出摊、检车、准备会\\
前站前一晚 & 提醒出发时间、打开铃声\\
前站 & \\
周日 & 前站执委会\\
\end{tabular}
}

正式周
\scalebox{1}{
\begin{tabular}{r |@{\foo} l}
周一 & 发布拉练报名帖\\
周二前& 联系宣传部安排拉练推送\\
周一-周四 & 预报名\\
周四 & 公布职务人员名单,拉群\\
周四-周五 & 职务人员培训和验收\\
周五 & 出摊、检车、准备会\\
正式前一晚 & 提醒出发时间、打开铃声\\
正式 & \\
周日 & 正式执委会\\
\end{tabular}
}

\paragraph{}

正式拉练结束后,\textbf{一周内}给卢老师发活动总结,\textbf{两周内}发布队长总结帖,\textbf{两周内}更新拉练帖,如遇考试周后两项可适当延迟,但不应晚于正式拉练结束后\textbf{四周内}。

\section{职务人员安排与培训}

\subsection{前站职务人员}

\subsubsection{人员构成}

前站一般不超过15人,去回程前旗助共4人、去回程后旗2人、押后1人、队医1人、拉秘1人、机动1人、队长1人、摄影1人、社考负责1人、午间主持1-2人。

部分拉练可能存在晚会负责、晚饭负责、住宿负责、游玩负责、午饭前站、物资负责等,视情况增加前站人员,但不应超过20人。

注意新老比、男女比,强度略大的拉练男女比略大于1。

注意前站{\color{red}不能出现情侣或其他类似关系}\footnote{如已分手的情侣、表白过的双方等暧昧关系。}!

\subsubsection{前站报名帖的注意事项}

必须包含以下内容:
\begin{itemize}[nosep,left=4em]
    \item 对前站的简要介绍(什么是前站);
    \item 强调报名前站不代表一定会被选上;
    \item 强调前站人员必须参加正式拉练;
    \item 路书号、路线概况,应能体现路线强度和特点。
\end{itemize}

\subsubsection{和职务人员的面谈}

\begin{itemize}[nosep,left=2em]
    \item 医学部和本部都要有;
    \item 尽量靠前安排,给自己排职务和职务培训留好时间;
    \item 初步考察报名会员对意向职务的熟悉程度,能达成初步共识的可以进行简单的职务培训;
    \item 即使现场人很多,最好一个一个来,自己实在聊不过来可以让随队帮忙,群聊效果并不好;
    \item 尽量不要现场答复具体职务,给自己做调整的空间;
    \item 聊天技巧:ID、年级院系、职务意愿和了解程度、骑行经历、社会经历。
\end{itemize}

\subsubsection{职务人员安排的原则}

\begin{itemize}[nosep,left=2em]
    \item 前旗前助至少有1个体力好的会员,后旗至少有2个体力好的会员;
    \item 前旗助认路,能够根据地图但不依赖导航判断行进方向,可提前确认前旗助,尽早展开培训;
    \item 拉秘应选择细致、耐心、变通的会员,同时周五下午出摊时需全程在场;
    \item 可以按照以下优先级考虑:性格外向、活泼、负责、热情>有冬游、暑期意愿>有通过冬训、春训考勤需求>低年级优先高年级。
    \item 综合考虑会员的学业压力、他能够投入的时间;
    \item 队医、押后人选\textbf{确定后}需询问实践部、队医组意见;
    \item 确定职务之前,给随队看一看。    
\end{itemize}

\subsubsection{重点职务人员培训与验收}

\paragraph{前旗助}

确定职务后,发送\href{https://chexie.net/bbs/content/?bid=7&tid=1062&p=1#2}{\textit{前旗助职务要求}}和\href{https://www.chexie.net/bbs/content/?bid=7&tid=879&p=1}{\textit{《当我们谈起前助时,我们在说些什么》}}
,确定职务培训时间(最好在周三之前);培训时可以让前旗助自己说一说自己需要做的前期准备工作、拉练时的时间线、如何规划留口和休息点、在休息点应该做什么、遇突发情况的处理办法等。在周五之前进行前旗助准备工作的验收,包括但不限于留口和休息点文档、前助讲话、路线识记等。

前站拉练结束后,队长和前旗助敲定休息点位置,并督促完善留口和休息点文档、前助讲话,如有较大的路线修改应另行验收。

休息点等可让前旗助自行寻找、判断,但队长应做到心中有数,效果不佳时及时干预。

\paragraph{拉秘}

发送\href{https://chexie.net/bbs/content/?bid=7&tid=1062&p=1#7}{\textit{拉秘职务要求}},提醒拉秘尽早找好接送站负责并让拉秘培训好接送站负责,根据实际情况判断是否需要队长对接送站负责进行进一步培训和确认。确定接送站负责后将接送站负责\textbf{拉进群里}。

及时将拉秘所需资料{\color{blue}\textit{2 拉秘打印资料}}\footnote{注意根据本次拉练做适当更改!}发给拉秘,并提醒拉秘提前打印相关资料、制作出摊表、签到表等。

提醒拉秘务必在周五16:30前上交个人责任书。

\paragraph{午间主持} 如有必要,可以对午间活动进行简要的彩排和审核。

\paragraph{其他职务}

具体职务内容请参见\href{https://chexie.net/bbs/content/?bid=7&tid=1062&p=1}{\textit{常规拉练职务培训}},可以将帖子直接发给相应职务人员,验收时让其简要概括。

\subsection{正式拉练职务人员}

\subsubsection{增设的职务人员}

检车负责1人、摄影多人、放坡负责1人、新的接送站负责各1人、增设的队医和押后(视拉练人数决定具体数目)、队医和押后助理、借车负责(迎新拉练)、爬坡前讲话(可由前助兼任,或另找老会员)。

\subsubsection{职务人员安排和培训的注意事项}

\begin{itemize}[nosep,left=2em]
    \item 检车负责:一般情况下找不承担拉练押后职务的押后,督促提前拉好押后群、分发检车标志,告知其需要检车的地点和时间,协商确定出摊检车标准。有陡坡的拉练应强调变速小档位不蹭;决定公路车能否过检和过检标准。
    \item 摄影:如找多人担任摄影,提前确定汇总照片、剪辑视频的会员。
    \item 放坡负责:可以找有经验的老会员,督促提前写稿,提前拉队首队尾群,放坡前要及时出现安排放坡
    \item 借车负责:热情、不社恐、认识的人多。
    \item 队医、押后和助理人选确定后需询问实践部、队医组意见。
    \item 爬坡前讲话:老会员即可。
\end{itemize}

\subsection{其他}

安排确定职务时,如报名帖中无该职务意愿,请\textbf{务必}提前和相应会员确认。

可以在职务人员培训时传达一些队长自己的想法,如对拉练的期待等。

及时讲解行者的使用,尤其注意队友位置共享。

\section{微信群的管理}
\textbf{注意}:\textit{微信群拉群时间应晚于论坛公布名单时间。}

\subsection{消息分级}

在群中发布的通知等内容,应按照重要程度分为收到请回复、群公告或@所有人、普通消息。重要的【时间】、【地点】应做特殊标记,并对消息的有效性做检查(如是否按照群公告更改群昵称等)。

将多件事情分项标号,避免多条短消息,尽量一条消息把应该表述的事情表述清楚。

\subsection{其他}

拉群后及时发布群公告,内容可参考{\color{blue}\textit{4 讲话内容}}。将路书号、职务安排、如何加入行者俱乐部及进行位置共享、职务人员通讯录一并发送到微信群中。

可适度水群,鼓励约饭越自习,适度进行团队建设。

\section{出摊、检车和准备会}

\subsection{出摊、检车}

队长及时将拉练通知发给拉秘,出摊时尽可能全程待在出摊地,解决非常规问题和新会员疑问,督促拉秘履职,尤其注意个人责任书应按时上交。

如当次拉练人数较多,应提前在实践部群中号召押后检车。

\subsection{准备会}

尽量精简,流程可参考{\color{blue}\textit{4 讲话内容}},注意前站不带旗。

\section{前站和正式拉练}

\subsection{前站拉练}

\subsubsection{起床集合}

起床后应在群中打卡,形式不限。\textbf{去程职务人员}集合时间应提前10-15分钟。

\textit{7:00出发,6:40没有发我起床了,就开始打电话,6:55没有见到人,就准备捞人。}

\subsubsection{出发前}

前助讲话,队医讲话,带热身,队长讲话。

如有职务人员迟到,视情况派送站会员捞人,但不要因此推迟出发时间过久,捞不到就算了。如其确有通过冬、春训考勤要求,可留老会员等待陪同追队,大队先行出发。缺失的职务人员可由暂时无职务的回程人员、老会员兼任。

\subsubsection{拉练中}

将前站作为正式大队的预演,记录时间线,观察各职务人员的工作情况,考察休息点、午间活动场地等。

\begin{itemize}[nosep,left=4em]
    \item 考察路况,有必要时在大队拉练做出路线修改、下车推行、提高车辆过检标准等要求;
    \item 考察休息点,不确定的休息点可以在备用休息点略作停留,由前旗助考察休息点情况;考虑休息点的载荷能力,卫生间、商店(不可兼得时优先卫生间),如果在较为繁华嘈杂的地区附近注意要能捞走;
    \item 考察午间活动场地,如有景区,可留下景区电话方便沟通;
    \item 重点关注前旗助和拉秘的工作状况,如前旗助对留口、休息点的考察,骑行手势和口号,对路线的熟悉程度等;拉秘是否能及时出现并进行签到。
\end{itemize}

\subsubsection{团队建设}

一方面要引导前站人员熟悉情况,在大队出行时起到引导其他会员的作用;另一方面也要注重小队成员的骑行状态,注重团队建设,提升前站人员的骑行体验。

\subsubsection{解散前}

所有人员到达终点后,签到、做放松、队长讲话、唱会歌、解散。鼓励约饭团建。

\subsubsection{解散后}

总结拉练当天的问题,职务人员的问题单独和其约谈,继续对职务人员进行培训(尤其是前旗助和拉秘)。提醒押后。队医更新日志,提醒摄影整理照片,提醒拉秘联系宣传部发拉练推送。

总结需要讨论的问题,在执委会上进行讨论。
\textbf{注意}:\textit{提交的执委会材料应尽量精简,将需要讨论的问题重点列出即可。}

前站结束后对路线、流程、休息点等的改动,应及时同步给前站人员。

\subsection{正式拉练}

正式拉练大体流程和前站相似,因此仅就不同处做特别说明。

\subsubsection{把控队伍节奏}

\begin{itemize}[nosep,left=2em]
    \item 告知职务人员遇到问题及时在职务群中告知队长;
    \item 确保后旗小分队有老会员,及时催促压后旗的会员、组织追队;
    \item 根据队伍到达休息点的分布、行者定位分布,并询问前旗助、老会员附近的情况、留口等待时长等,判断队伍大致情况,和前旗助协商休息时间和后续骑行速度;
    \item 队长应尽量骑行在队伍中后位置,关注落后的会员和大队整体状况。
\end{itemize}

\subsubsection{先遣队和前站}

如设有先遣队、前站,应提前明确出发时间,和大队至少相差15分钟,提前告知拉秘进行签到,并提前告知押后、队医配备简易清创、补胎工具。

先遣队一般少休息或不休息,只在适当位置补给,因此应让体力较好且参与过拉练的会员担任。先遣队应设立一名先遣队队长,并设立前后旗带路收尾,保持队伍紧凑。

\subsubsection{追离队}

\begin{itemize}[nosep,left=2em]
    \item 除不可抗力(包括学业原因在内),不允许任何人追离队,审批从严,慎重考虑;
    \item 一般情况下,允许追离队的拉练为凤凰岭正式、慕田峪正式、黄花城正式。
    \item 可以追离队的会员范围应参考如下标准:
    \begin{itemize}[nosep]
        \item 正在担任或担任过部长(非当车协年新会员副部)、主席、理事、车队队长;
        \item 参加过暑期远征的会员(已经成团但是没去过的不算);
        \item 从未参选暑期的,或者未能成行,但是参加协会各类活动多次,频繁参加活动时长在一年以上,并参与过重大活动的组织,并扮演比较重要的角色,得到大家一致认可的。
    \end{itemize}
    \item 在确定所有追离队信息后,方可向队长申请追离队,信息变更后需\textbf{重新申请};
    \item \textbf{保险}\ 对于慕田峪、黄花城这类拉练,即使追离队只骑一天/坐车一天,但不能保证行程不会因为一些因素延长到两天。追离队人员统一购买两天保险;
    \item \textbf{个人责任书}\ 参加活动就需要签署个人责任书;
    \item \textbf{缴费}\ 统一收取全部费用,多的结束后再退;
    \item \textbf{检车}\ 只去去程,回程运回来的车在坡顶不需要检,提前在追离队群中统计并告知检车负责和拉秘;
    \item 为使得信息通畅,可以建立队长、追离队负责、拉秘、食宿负责、机动的群,关于人员变动信息方便及时共享。
\end{itemize}

\subsubsection{其他细节问题}

\begin{itemize}[nosep,left=2em]
    \item 队长讲话后出发时间的制定,根据队伍情况,如大部分人还没有收拾好,可适当多留2分钟,仓促出发反而状态不佳;
    \item 提醒前助携带大喇叭;
    \item 提醒拉秘紧跟前旗,到达签到点后及时组织签到;
    \item 午间休息督促检车负责组织检车;
    \item 强调不要离开休息点、坡顶过远;
    \item 拉练结束后提醒拉秘及时补收或退费并发出财务总结、队医及时更新队医日志、押后及时更新押后日志、社考及时发出社考帖(前述工作应在一周内完成)、摄影及时将照片上传网盘并分享给大家。
\end{itemize}

\section{意外情况的处理}


\subsection{基本原则与预案}
\begin{itemize}[nosep,left=2em]
    \item \textbf{原则}:大队大于局部,安全大于一切;
    \item 提前参考队长总结和执委会记录,做好一定预案;
    \item 对紧急情况心中有数,做好部署(明确告诉当事人应该怎么做),在确定相关人员身体无大碍的情况下,应找可信任的老会员在原地处理事故,队长随大队继续前进;
    \item 若发现队伍的异常情况,可休息点出发前等增加队长讲话,强调相关事项(如丢人,强调留口,不跟前旗,强调尽量跟前旗);
    \item 出现问题并解决后,及时反思,想清楚是哪里没做好、是不是自己的问题,不要有过大的压力,\textbf{出问题并不代表队长做的不好};
\end{itemize}

\subsection{后旗落后过远的处理}

如后旗出现需要较长处理时间的事故,应电话告知后旗小分队附近的老会员收留口,告知队中押后、队医沉到队尾(遇到收留口的老会员),充当临时后旗小分队。

队伍到达休息点后,如队伍紧凑,前后旗相继到达,休息15-20分钟即可。如后旗处理事故拖后较远:
\begin{itemize}[nosep,left=2em]
    \item 如已有临时后旗收留口,按临时后旗到达时间休息15-20分钟,并告知后旗小分队自行追队、休息;
    \item 如尚未组织临时后旗收留口,可以组织后旗小分队附近老会员充当临时后旗收留口,在临时后旗到达休息点后,若后旗仍落后较远,处理同前。
\end{itemize}

到达终点时,根据行者定位,若后旗落后较远,可根据实际天气情况决定等待后旗或由队医先带放松。

在得知后旗状况较差或在爬坡前,可以鼓励(利用私聊、爬坡前讲话等)体力好的老会员说留在后旗附近帮助后旗。

\subsection{恶劣天气情况下的处理}

在拉练出发前,队长应特别关注天气预报,及时了解拉练沿途的天气情况。如有特别恶劣的天气情况可能发生,应做好相关预案。

\subsubsection{雨雪和污染天气}

如有小雨、雪天气情况可能发生,应及时提醒参与拉练的会员天气情况,准备保暖衣物和雨具,在办理出摊手续时\textbf{确保所有人}具备雨具。拉练当天可视天气情况调整拉练终点。

如有中雨、雪及以上级别的天气情况、中度污染、沙尘暴等极端天气情况可能发生,请及时联系文体部、主席团。如为正式拉练,根据天气情况可以选择拉练改期或取消;如为前站拉练,可以灵活选择交通方式进行前站,如只带领去回程前旗助做地铁打前站等。

\subsubsection{高温天气}

\begin{itemize}[nosep,left=2em]
    \item 视情况调整拉练终点和强度;
    \item 提醒队医多携带一些中暑药物、补液盐,为可能发生的中暑、抽筋做好准备;
    \item 提醒押后检车时不要给车辆补气过多,并在出发前适当放气降低胎压;
    \item 及时提醒参加拉练的会员,如有不适及时停车请求身边人的帮助。
\end{itemize}

\subsection{人员与车辆问题的处理}
\begin{itemize}[nosep,left=2em]
    \item 如有会员因坏车、体力不支等情况需要坐车,安排无职务的老会员(优先)或机动陪同坐车;
    \item 如留口汇报丢人,在合适的地方(休息点)告知前旗助和拉秘,增加一次签到;如留口汇报时附近有老会员,可视情况安排老会员去追,但队长务必把控好队伍整体情况;
    \item 如有需要牵车的情况,至少安排两名会牵车的会员,相互照应,交替牵车;
    \item 若发现队伍的异常情况,可休息点出发前等增加队长讲话,强调相关事项(如丢人,强调留口,不跟前旗,强调尽量跟前旗);
    \item 出现问题并解决后,及时反思,想清楚是哪里没做好、是不是自己的问题,不要有过大的压力,\textbf{出问题并不代表队长做的不好};
\end{itemize}

\subsection{交通事故的处理}

队伍遇有交通事故,队长在事故发生地点附近的,可及时前往,初步了解情况,寻找队伍附近可靠的老会员(至少两人)协助解决问题,队长及时追队,保证大队的正常行进。事故比较严重的,可以让大队先走,处理事故的会员和事故相关方自行返校。

\subsubsection{基本原则}
\begin{itemize}[nosep,left=2em]
    \item 避免拉练队员围观,及时疏导交通,不能耽误队伍进程;
    \item 保持冷静不等于策略上不采用吵架的方式;
    \item 绝大多数情况下,对方的挑衅、威胁不用在意;
    \item 在交通事故、拦路设卡要钱、社会纠纷中,对方往往提高音量,得理不饶人;
    \item 在一定情况之下,用较为强硬的语气和姿势来表现我方的强势和坚持,并不代表队长内心当中已经失去冷静。其他会员更不可助他人威风;
    \item 除非队长或者负责处理事故的会员已经明显失去冷静,否则其他会员不要在边上说“冷静、冷静”,越说越烦。
\end{itemize}

\subsubsection{明确责任和主张}

\begin{itemize}[nosep,left=2em]
    \item 发生事故后,处理事故的会员应首先和我方事故相关会员(当事人、其他看到事故现场的会员)单独了解情况,初步明确情况,判断己方责任;
    \item 我方明显过错,先向对方承认我方错误,尽量争取减少赔偿或者不赔偿。根据对方的态度,善于卖弱卖惨,态度平和和对方商量,对方很可能让步,及时脱身离开;若对方索要明显和实际情况不符的赔偿并拒不让步的,及时报警;
    \item 对方明显过错,则应当强调我方的立场,造成人车等财产损失的,索要相应的赔偿。对方明确表示不赔偿或者甚至反咬一口,及时取证报警,等待警察前来。在这个过程中采取反威胁,也有可能问题和解;
    \item 确实责任不明的,耐心等待警察前来调解;
    \item 遇到危险情况,首先保证自身生命财产安全,可视情况及时逃离现场并报警。
\end{itemize}

\subsection{严重伤人事故的处理}

请参考\href{https://www.chexie.net/bbs/content/?bid=7&tid=839#5}{\textit{意外事故的处理案例}}。及时向主席团、理事会汇报。


\center\Large{\color{red}\textbf{队长加油!}}

\end{document}

