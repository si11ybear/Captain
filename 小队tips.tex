\documentclass[UTF8]{ctexart}
\usepackage{amsmath, amsthm, amssymb, graphicx, amsfonts, indentfirst}
\usepackage{fancyhdr,color,framed,enumitem}
\usepackage[perpage]{footmisc}
\linespread{1.5}
\usepackage[letterpaper,top=2cm,bottom=2cm,left=3cm,right=3cm,marginparwidth=1.75cm]{geometry}
\usepackage[colorlinks,
            linkcolor=blue,
            anchorcolor=blue,
            citecolor=blue]{hyperref}

\ctexset{tocdepth=3}
%\ctexset{subsubsection = {number=\arabic{subsubsection}}}       %subsubsection形式改为1,2,3,...
\title{小队队长培训}
\author{小笨熊}
\setlength{\parindent}{4pt}
\setlength{\headheight}{27pt}
\pagestyle{fancy}
\newcommand{\foo}{\hspace{-2.7pt}$\bullet$ \hspace{5pt}}
\begin{document}
\maketitle

\tableofcontents

\clearpage
\section{小队多日拉练}

\subsection{小队多日拉练的特点}

小队多日拉练更强调团队和骑行体验,由于人数较少,安排上更为灵活,但每个人所占的比重增加,更需要关注到队伍中每一个成员。为此,小队通常可以安排更多的团建项目,多约饭约自习,增进队员之间的情感,利用国王与天使等活动,提升拉练途中的体验感。

\subsection{前期准备}

\begin{itemize}[nosep,left=2em]
    \item 准备会前通过问卷收集职务意愿,规划首日职务,并和会员确认;
    \item 出京拉练需要额外的报备流程,尽早完成;
    \item 在出发前安排至少两次准备会,第一次准备会相互认识、破冰,简要介绍路线、日程、职务意愿,第二次准备会同大队;
    \item 尽早确定拉秘,安排出摊、检车、借头盔、收费等事项,一般的标准是早饭8元、午晚饭15元、住宿25元,并留20-30元机动;
    \item 及时确认押后负责和队医负责,制作车况问卷、身体状况问卷,根据实际情况准备押后包和队医箱,为分房提供参考(睡眠质量和是否发出异响)。
\end{itemize}

\subsection{职务安排和变化}

\subsubsection{职务安排的原则}

多日拉练中,尽量不要让一名会员连续处于前后旗位置或连续打前站,应让每名会员都有在队伍\textbf{中}骑行的体验。在安排职务时,可以提前公布第一天职务,并提前安排好第二天职务并职务培训,但先不公布,根据第一天骑行状况,待第一天组会后再行公布。

\subsubsection{职务变化}

\begin{itemize}[nosep,left=2em]
    \item \textbf{前旗助}\  前旗助可以根据队伍情况适当调整速度(通常是提速),无需制作留口和休息点文档,对路线适度熟悉即可。同时小队尽量减少留口,强调队伍紧凑,被红绿灯等分割队伍后前旗及时减速,等待队伍;如有扎胎等情况,可临时找合适的地点停车等待后旗。
    \item \textbf{前站}\  由队长确定餐标,及时沟通前站情况,前站不顺时告知前旗助减速,携带建议清创和补胎工具。一般采取老带新的模式,第二次可以让被带的新会员带其他新会员。早饭前站通常由拉练中违反纪律的会员担任,若无,则由当日\textbf{前旗}兼任。
    \item \textbf{押后负责}\  由师父培训,制作车况问卷,根据问卷结果收押后包,分配组内押后和车辆对应关系,组织检车。
    \item \textbf{队医负责}\  由队医组培训,制作身体状况问卷,根据问卷结果收队医箱,组织查房。
    \item \textbf{司乐}\  提前收集歌单,视情况安排,可有可无。
    \item \textbf{玩乐负责}\  提前搜集游玩地信息,视情况提前购票。
    \item \textbf{邮局负责}\  提前探查邮筒位置,准备适量邮票,可有可无。
    \item 根据实际情况,可设置押后旗、队医旗、前旗助等合并职务。
\end{itemize}

\subsubsection{组会}

根据实际情况,队长应对组会时长做出预估,提前提醒发言组员控制发言时长;如时间较充裕,队长可以提出一个话题,让组员多聊一聊。时长不宜超过1.5小时,最多不要超过2小时。

鼓励组员说出自己的真实感受和体验(如职务体验、开心和不开心的事情、骑行感受等),不要过于强调在组会中解决问题,对于纪律、安全、团队意识的问题,队长可在组员发言后做简要提醒和补充,确保组会大方向的正确。

\subsubsection{第一天晚上的时间安排}

\begin{itemize}[nosep,left=2em]
    \item 在到达住宿点时间不可能更早的情况下,应和食宿前站做好沟通,确保队伍到达晚饭点就能尽快吃饭;
    \item 给押后检车、队医查房预留时间,可以利用组会时间进行按摩等;
    \item 在组会结束后,队长应尽快将第二日职务安排告知组员,根据实际情况可以略作调整;
    \item 应至少预留6小时睡眠时间,保证押后、队医的休息,同时将出发时间等发到群里,以便检车押后、查房队医知晓;
    \item 可以制定出发时间,不限制起床时间,让组员根据个人情况自行决定;
    \item 如时间富裕,可安排一些团建活动或自由活动,但应告知组员结伴出行、提前告知队长,不要单独外出。
\end{itemize}

\subsubsection{骑行节奏}

\begin{itemize}[nosep,left=2em]
    \item 如队伍状况整体良好,可适当延长两休息点间的距离、提高骑行速度;反之则应控制前旗助速度,适当延长休息时间,调整队伍状态;
    \item 根据实际情况可以多增加一些拍照点,增进团队氛围;
    \item 在可控的前提下,不必太过强求早到终点;
    \item 根据天气和实际情况,可以适当破风、推人。
\end{itemize}
\center\Large{\color{red}\textbf{队长加油!}}

\end{document}

