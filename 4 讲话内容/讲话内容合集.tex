\documentclass[UTF8]{ctexart}
\usepackage{amsmath, amsthm, amssymb, graphicx, amsfonts, indentfirst}
\usepackage{fancyhdr,color,framed,enumitem}
\usepackage[perpage]{footmisc}
\linespread{1.5}
\usepackage[letterpaper,top=2cm,bottom=2cm,left=3cm,right=3cm,marginparwidth=1.75cm]{geometry}
\usepackage[colorlinks,
            linkcolor=blue,
            anchorcolor=blue,
            citecolor=blue]{hyperref}

%\ctexset{subsubsection = {number=\arabic{subsubsection}}}       %subsubsection形式改为1,2,3,...
\title{队长讲话合集}
\author{小笨熊}
\setlength{\parindent}{4pt}
\setlength{\headheight}{27pt}
\pagestyle{fancy}
\newtheorem{theorem}{\indent 定理}[subsection]
\newtheorem{definition}[theorem]{\indent 定义}
\newtheorem{prop}[theorem]{\indent 命题}


\begin{document}
\maketitle

\tableofcontents

\section{写在前面}

讲话模板是为了提醒队长不要遗漏需要强调的重点,最好不要照搬模板里的内容,尤其是解散前队长讲话,每个人大概率只会带一次队,你想说的和其他参与拉练的会员想听的一定不是千篇一律的模板。

\section[小队拉练]{小队拉练 \protect\footnote{双日、前站等。} }

\subsection{建群后群公告}

欢迎大家来到双日A2组!

我是队长小笨熊。

在一周后的周末,我们将再一次踏上征程,前往琉璃庙。在此之前,有几件事情需要通知一下!请大家仔细阅读。

\begin{enumerate}[nosep,left=4em]
    \item 请大家将自己的群名片改为【id 手机号】,如:小笨熊 19931109246。
    \item 请大家于【今天22:00】前填写末尾的问卷,在确定职务时队长会再次确认。    
    \item 预计在4月18日晚5:30进行第一次准备会和约饭,地点暂定于家园3层室外,请所有人员参加,如若不能参加,请在问卷中填写尽可能多的可以和队长约饭的时间~    
    \item 收到请回复一个【小奶龙】表情包。
\end{enumerate}


\subsection{出发前一晚微信群提醒}\label{小队}

提醒大家几件事:
\begin{enumerate}[nosep,left=4em]
    \item 明天可能会下雨,请大家带好雨衣。
    \item 今晚早些休息,明日骑行强度较大,建议入睡时间为23:00之前。集合时间为6:10,本部会员请在5:45之前/其他会员请在5:30之前在群里发:我起床了。
    \item 仔细对照拉练通知的物资清单,把手电、手机、移动电源、码表(如有)等充满电,并在【今晚】把所有物资收拾好,不要等到明天早上。
    \item 今晚入睡前【务必打开手机铃声】,定好闹钟,确保明天可以打通且听见电话。
    \item 明天第一个休息点有麦当劳,但不会预留过长时间。所以还请根据自己的情况在出发前吃适量早餐或提前点餐。
\end{enumerate}

希望明后两天的时光和大家过的开心~

\subsection{出发前队长讲话}

大家好,我是本次双日A2组琉璃庙拉练的队长,我的id是小笨熊。现在,请大家拿出手机,打开手机铃声,把声音调到最大,确保能够及时接听任何一位队友的电话。

在路上,身体不适找队医,车有问题找押后,其他任何问题都可以和我打电话(,我的手机号是xxxxxxxxxxx)。

接下来,我强调三件事情:

第一,是安全意识。请大家务必遵守交通规则、遵守协会出行纪律,确保我们能够平安归来。违反相关规定的将按例罚跑,累计罚跑超过11km将取消暑期预备队员资格!

第二,是时间观念。请大家严格遵守前助、队长定下的时间点,不要拖延,你耽误的一分钟,可能是夜路多走的五分钟,都是小队十五个人少休息的十分钟!

第三,是团队意识。希望大家积极陪骑,和身旁的人多交流,多了解,在任何时候,我都不希望看到有人落单!

好了,我要说的就是这些,希望这次双日拉练一路顺利,平安归来!现在是x点xx分,大家收拾一下,我们x点xx分(约3分钟后)出发。

\subsection{解散前队长讲话}

首先 我想感谢每一个人(可以说明每一个职务人员,注意包括没有参加拉练的职务人员\footnote{如接送站负责、借车负责等。}。

其次 表扬大家的团队意识、时间意识、安全意识等,如果有做的不好的也可以指出。

再次 表扬一些为团队做出特殊贡献的会员(如牵车推人、押后检车到很晚、体力不好但自己坚持骑下来等等一切你觉得值得一提的事情。

最后,我想跟大家说一句话:前路漫漫,愿我们顶峰再会!

大家一起唱一下会歌~

\textbf{注意} \ \ \textit{解散前讲话,尤其对于小队拉练的解散前讲话。不建议出发前就写好稿,可以提前勾勒大致框架,在回程路上慢慢填充细节,有真情实感的解散前讲话才是有意义的解散前讲话。}

\subsection{非前站拉练结束当晚在微信群中}

感谢大家参加双日拉练,大家今天都很棒,相信的大家会记住这个独一无二的周末的!最后还有几句话要提醒一下大家:

骑车脱水比较多,大家睡前要多补一些水,以防晚上渴醒,睡觉的时候可以把腿抬高。

辛苦押后队医及时更新日志,拉秘及时更新财务总结。

大家及时更新状态,明天还要上学!(虽然我要自然醒了

感谢我们每一个人,大家真的牛!我想说的话很多,但这里写不下了,所以欢迎【大家】写足音!

\subsection[前站拉练结束当晚在微信群中]{前站拉练结束当晚在微信群中\protect\footnote{感谢22禅房队长\ 华年|余博涵。}}

首先,还是非常感谢大家参加禅房前站,这次前站能遇到你们,我感到非常幸运。说实话,这次前站的顺利程度其实超出了我的预期,虽然有一部分客观原因,但是这也和大家的认真准备和对纪律的重视是分不开的。大家的前期准备过程我都看到了,你们真的辛苦了!希望正式拉练时,我们前站的每一个人都能担负起小队长的身份,像本次前站一样守时守纪,为其他人做一个表率

不过,大家也应该意识到,前站并不能暴露出正式拉练时可能发生的所有问题,因此请大家保持对自己职责的认真态度。现在距离正式拉练还有一周半的时间,希望大家在此期间能回顾一下前站时自己做的事情,想一想还有哪些事情可以做得更好。同时,也请大家在正式拉练之前重新想一想自己的职责内容。在此期间,队长也会陆续和大家聊一聊,补充一些细节

最后,预祝一周半后的禅房正式顺利!我们一起努力!

\textbf{注意} \ \ \textit{有集体性问题应直接指出,职务问题可以私聊相关职务人员。}

\section{大队拉练}

\subsection[职务人员群建群后群公告]{职务人员群建群后群公告\protect\footnote{感谢23高大东队长\ 土山|禹坤煜,有改动。}}

感谢大家报名23高大东拉练的职务人员,为大队的出行保驾护航,我是本次拉练队长禹坤煜|土山。接下来有一些事情需要大家一起完成:

\begin{enumerate}[nosep,left=4em]
    \item 请将群备注改为ID|姓名|职务
    \item 请将所有职务人员的手机号存入通讯录(可利用附件中的vcf文件)。
    \item 请大家参加今天【17:10】在出摊地旁边的准备会,如需请假,请在今天【12:00】前私戳队长。要求准备会前,去程前后旗取旗,去程前助取3个防潮垫,去程押后、队医收好押后包、队医箱,所有前后旗学会绑旗。准备会上将进行自我介绍,路线介绍,电话号码抽查,拉练流程介绍。
    \item 本次拉练路线已上传行者app,去程路书\#3675748,回程路书\#3664962,请大家确保已经加入“CAPU行者”俱乐部
    \item 明天职务人员集合时间为【5:50】,其他人员时间为【6:00】,预计【6:25】前出发,地点为北大东门停车场,请勿迟到。
    \item 请大家在出发前再次熟悉自己的职务内容以及队长私戳告知的注意事项,确保明天能够高质量地完成职务内容。
\end{enumerate}

另外需要提醒一下【去程队医】,收到请回复:
\begin{enumerate}[nosep,left=4em]
    \item 本次拉练气温较高,易多发中暑、抽筋等情况,请携带足够的中暑药、补液盐(约20人份)。
    \item 山中蚊虫较多,易发生蚊虫叮咬的情况,请做好准备。
    \item 本次拉练中前旗充当坡顶计时员,需一前站包,并准备好防中暑、防抽筋物资。
\end{enumerate}

以及【去程押后】,收到请回复:
\begin{enumerate}[nosep,left=4em]
    \item 需准备一前站包,注意多携带一小打气筒和备胎,尺寸在今天出摊时询问飞舟与峰林。
\end{enumerate}

最后,希望大家能好好准备,共同努力~希望明天的拉练顺利进行!

\subsection[准备会]{准备会\protect\footnote{感谢21文体部部长、21潭柘寺队长\ 霖|刘雨霖,有改动。}}

大家先进行一下自我介绍。

我在这里想要强调三点:

\begin{itemize}[nosep,left=4em]
    \item 首先是纪律意识。因为我们是最后一次大队拉练,所以我们的风格很大程度是会影响到往后的双日,穿越,体测的风格,必须要在纪律上做好表率。具体体现在讲话、热身、放松时候不要说话,到点就出发等等。我们是职务人员,对大队纪律有着至关重要的影响,希望大家能做好表率。
    \item 然后是时间意识。希望大家在出发前不要迟到。我们是6点40分集合完毕,希望大家在今天晚上定好闹钟,打开手机铃声。明天6点20分前没有在职务人员群里发起床了的我会直接打电话,电话不接就宿舍找人。迟到的话直接开罚跑。同样,希望大家在休息点时不要磨蹭及时出发。
    \item 最后是团队意识。我们作为职务人员站,就是一个共同努力带好大队的团体,希望大家能在路上互相帮助。
\end{itemize}

接下来:
\begin{enumerate}[nosep,left=4em]
    \item 请去回程前助来介绍一下本次路况。
    \item 请大家在拉练路上全程打开行者,确保我能看到所有人的位置。出了任何情况请及时向队长报告。
    \item 抽查联系方式。
    \item 检查是否收好押后包、队医箱、防潮垫和前后旗。
\end{enumerate}

好,准备会就开到这里,大家一起加油~

\subsection{出发前一晚职务人员群提醒}

同小队拉练\ref{小队}。

\subsection[出发前一晚大水群提醒]{出发前一晚大水群提醒\protect\footnote{感谢21文体部部长、21潭柘寺队长\ 霖|刘雨霖。}}

明天要参加潭柘寺的朋友们,再次提醒一下:
\begin{enumerate}[nosep,left=4em]
    \item 明天天气多云,-4°-7°,注意保暖,建议穿衣模式:速干-(薄毛衣)-抓绒-冲锋衣,穿上【厚裤子】、【厚袜子】和厚鞋子,不要露出脚踝,带上【长指厚手套】,如果有,戴上头巾、耳罩和护膝。带上保温杯,准备好午饭。
    \item 今天早点休息(建议入睡时间23:00前),收拾好驮包或者背包,【手机铃声开到最大,定好闹钟】,手机和充电宝充满电。
    \item 明天【7:00】之前在本部东门停车场外集合完毕,并找拉秘签到完毕,迟于7:01记迟到,将按照条例罚跑。建议到达时间早于【6:55】。
    \item 入寺参观的会员:门票在明天午间时提醒大家购买,不需要现在购买。
    \item 关于路线:请下载行者app,搜索路书:去程\#2912024 回程\#2913359。
\end{enumerate}

\subsection[出发前队长讲话]{出发前队长讲话\protect\footnote{感谢21文体部部长、21潭柘寺队长\ 霖|刘雨霖,有改动。}}

大家好,是本次潭柘寺拉练的队长,我的ID是霖。从现在到回程返回北大、我宣布解散之前,由我负责本次拉练,请大家服从我的安排和决定。

请大家拿出手机记一下我的手机号xxxxxxxxxxx,再说一遍,我的手机号是xxxxxxxxxxx。

请大家把手机的铃声开到最大,保证通讯畅通。路上身体不适找队医,车有问题找押后,其他事情都可以给我打电话!

接下来我要强调3点:
\begin{enumerate}[nosep,left=4em]
    \item 安全意识,这是最最最最重要的一点。我们一定要平平安安的去,平平安安的回,遵守交通规则,遵守协会出行纪律。我在这里强调5点:严禁超前旗、严禁双手离把、严禁下坡超车、严禁闯红灯、骑行时必须佩戴头盔!我再说一遍!严禁超前旗、严禁双手离把、严禁下坡超车、严禁闯红灯、骑行时必须佩戴头盔!违反者,将罚跑3-5km米。罚跑并不是一件光彩的事,并且罚跑累计11000m不允许参加当年的暑期。请大家随时拥有安全意识,时刻保持警惕。
    \item 时间观念。潭柘寺拉练的收队点和休息点较多,请大家严格遵守前助定下的出发时间,保证到出发时间能够跨上自行车立即出发,不要拖沓,你拖延1的分钟不仅是你一个人的1分钟,是整个大队80个人的1分钟。本次拉练,共设由5次签到:早晨签到,戒台寺分组放坡时签到,到达午间点签到,潭柘寺分组放坡时签到,返回北大停车场签到。请大家记住自己的编号,主动告诉拉秘。
    \item 团队意识。刚才前助有介绍留口和骑行手势,我再强调一下,希望体力较好的会员能够积极当留口,留口是一件很光荣的事情,保障大队的行进、为大队指路。并且请积极传递骑行手势,单手控把不稳可以喊出来。希望大家能够相互配骑,结伴骑行,尤其是在爬坡和夜路的时候不要让任何人落单。
\end{enumerate}

最后总结:注意安全!时间观念!团队意识!然后,就是祝愿大家能够享受本次拉练~

现在是x点xx分,大家现在回去收拾一下东西,准备跟前旗列队出发!我们x点xx分(约3分钟后)出发。

\subsection[解散前队长讲话]{解散前队长讲话\protect\footnote{感谢21文体部部长、21潭柘寺队长\ 霖|刘雨霖,有改动。}}

大家骑了一天应该都比较累了,我就尽量长话短说,让大家能够尽早吃饭和休息。

今天的拉练算是冬训强度最大的一次,共有3个坡,大家的表现都非常非常的棒,首先给自己鼓个掌叭。

关于拉练中的表现,我想要单独的提一下/夸一下,首先是大家都很有时间观念,才让我们能够在6:20到达学校,也很有团队精神,今天爬坡时我没有看到任何人落单,夜路也保持比较紧凑的队伍,大家骑行口号的传递也很好。这一点我希望能够在之后的拉练、冬游、暑期路上保持和注意。

迟到算是一个比较严重的纪律问题,会耽误大队的进程,迟到的同学请自觉查看论坛工作区的罚跑执行方式,主动联系理事完成罚跑。因为每一个同学迟到的一分钟,都延误了大队的进程,都让大队多走了1分钟的夜路,增加了一份危险,所以希望之后大家能够严格遵守时间观念。

潭柘寺是最后一次冬训的拉练了,之后马上就要冬游了,我希望大家能够记住,所有你爬过的坡、陪你爬坡的人,对你喊过加油的人、提醒你急弯减速的人、为你指路的人、给你传递骑行手势的人,在坡顶分过的零食、饼干、火锅,参观过的寺庙和玩的午间游戏,精美的手绘明信片,希望这些都能成为你关于21潭柘寺的回忆,成为关于协会的回忆,并将这种感受,这种对自我超越、这种团队的精神带到冬游、带到之后的拉练、带到暑期。也祝愿大家冬游顺利,用潭柘寺开启一段新的旅程。

最后,大家一起唱一下会歌~

\textbf{注意} \ \ \textit{可以增加感谢职务人员环节,但请不要忘记接送站负责、借车负责和所有来接送站的伙伴们!可以提前勾勒大致框架,慢慢填充细节,拉练带完了,一定有很多自己的话想表达~}

\subsection[拉练结束当晚在职务人员群中]{拉练结束当晚在职务人员群中\protect\footnote{感谢22禅房队长\ 华年|余博涵。}}

首先,真的非常感谢大家的支持,感谢大家一路的陪伴。在昨天的拉练中,我深深地感受到,每一位职务人员都像是分布在大队中的点点星光,点燃自己,照亮了整支队伍。不得不说,大家真的过于优秀了,多亏了大家的精心准备和辛勤付出,队长成功当了一整天咸鱼x,在这里好好地夸一夸大家~

虽然这一天的时间过得很快,对于身经百战的各位来说,可能也只是一次普通的拉练,但对于队长而言,这也许是一种奇妙的缘分,能在这次拉练和各位相遇,我感到十分的幸运。感谢的话实在太多了,在这里就不多说啦,之后在队长总结中再一一细说吧!

最后,要是大家觉得这次拉练有啥问题,欢迎私戳队长指出哦~

\textbf{注意} \ \ \textit{可以提前勾勒大致框架,慢慢填充细节,拉练带完了,一定有很多自己的话想表达~}


\center\Large{\color{red}队长加油!}

\end{document}

